\chapter{Shrnutí}

\section{Závěrečné zhodnocení aplikace}

Zhodnocení aplikace z pohledu splnění požadavků vychází ze srovnání user stories v části týkající se metodiky a nesplněných user stories uvedených v kapitole 6 - testování \ref{tab:failed-accept-tests}. Celkový počet user stories je 29, z toho 4 nebyly naplněny. Procentuální splnění požadavků se tak dostane na 86\%.

\section{Doporučení do dalšího vývoje}

Zátěžové testy ukázaly, že aplikace se stává poměrně pomalou s přibývajícími úkoly v jednotlivých projektech. Problém tkví ve čtení dat z databáze. Řešením by mohla být nějaká vyrovnávací paměť (cache), která by byla umístěna mezi databází a zbytkem aplikace. To s~sebou nutně přinese mnoho dalších komplikací a zrychlení aplikace tak může být poměrně drahé. Mezi největší problémy patří určitě volba úložiště vyrovnávací paměti a také její invalidace (smazání neaktuálních dat). Úložiště musí být dostatečně rychle přístupné, aby vůbec mělo smysl vyrovnávací paměť implementovat. Nasnadě je využití souborové cache, ale čtení dat z filesystému nemusí být zrovna nejrychlejší. Lepší by bylo ukládání dat do operační paměti, jenže k té nemá JavaScript přístup. Dalším problémem je invalidace cache, tedy odstranění neaktuálních dat z paměti. Je totiž poměrně velký problém určit, kdy už data nejsou aktuální.

Další funkce, na které se přišlo během vývoje a které byly zařazeny do kategorie \uv{hezké mít} (angl. nice-to-have) jsou tyto:

\begin{itemize}
\item sledování i cizích projektů
\item vytváření vlastních přehledů úkolů (kombinace štítků a projektů)
\item automatické sledování commitů do synchronizovaného repozitáře
\item modulárnější vizuální stránka s využitím knihovny MustacheJS\cite{mustache}
\end{itemize}

Tyto funkce se ukázaly buď jako obtížně realizovatelné (vyrovnávací paměť, vytváření vlastních přehledů) nebo poměrně zbytečné (MustacheJS, sledování commitů), ale přesto byly uloženy, aby se na ně nezapomnělo.

\subsection{Práce s IDE Titanium Studio}

Součástí zadání této práce byla i rešerše Titanium Studia. Jejím cílem bylo ověřit možnost použití této platformy k vývoji desktopových aplikací. Jedna strana jsou oficiální dokumentace a druhá její reálné používání, jak se ukázalo mnohokrát během vývoje. Ve fázi testování se ale ukázalo, že spoustě problémů se lze vyhnout tím, že se vývoj provádí v operačním systému Windows a ne na Linuxu nebo Macu, kde už jen samotné zprovoznění IDE se ukázalo jako problém.

Nicméně Titanium API poskytuje spousty funkcí, které by se jinak musely doprogramovávat ručně. Není tak nutné zahrnovat do aplikace skripty napsané v jiném jazyce (Titanium Studio podporuje vývoj v PHP, Ruby a Pythonu), ale je možné využít připravené rozhraní. Těžko si například představit, jak náročné by bylo vytvoření spojení s databází jen pomocí JavaScriptu.

Existuje však jeden velmi výrazný nedostatek - vychytávání chyb (debugging). To je v~tomto IDE velmi špatně zpracované. Autoři Titanium Studia sice nabízí rozšířený editor, který by měl mít debugging zpracovaný lépe, ale ten už není poskytováno zdarma, a to ani ke studijním účelům. Ve většině případů je tak vývojář na metodu pokus-omyl, kdy i oprava banálního překlepu může trvat velmi dlouho. Během vývoje této aplikace sice došlo k několika aktualizacím a IDE tak hlásí aspoň některé chyby, ale ve spoustě případů prostě jen zamrzne a neposkytne vůbec žádnou zpětnou vazbu o tom, co a kde se vlastně stalo.

Na základě získaných zkušeností lze Titanium Studio doporučit dalším vývojářům, kteří by chtěli vyvíjet desktopové aplikace v jiném jazyce než v Javě nebo C\#. V nejčtenějším IT magazínu v ČR Programuje.com jsem publikoval článek\cite{selfpromo}, představující Titanium Studio a práci s ním. Dle reakcí čtenářů lze usoudit, že platforma má před sebou budoucnost a má smysl ji dále rozvíjet.

\subsection{Psaní aplikace zcela v JavaScriptu}

Další teorií, kterou měla tato aplikace za cíl potvrdit nebo vyvrátit, byla otázka, zda je možné vytvořit aplikaci zcela v JavaScriptu bez pomoci dalších programovacích jazyků. Ukázalo se, že tento postup je možný, ale zahrnuje dost problémů a kompromisů. Není například možné vytvářet rozhraní (interface) ve smyslu Javy nebo i PHP. To samé se týká abstraktních tříd. Rozšiřování aplikace o další moduly tudíž není tak snadné, jak by mohlo teoreticky být.

Dalším problémem je to, že JavaScript není primárně objektový jazyk, ale spíš procedurální, a některé konstrukce se vytváří hodně neohrabaně. Problém, který se často objevoval, je ztráta kontextu objektu. Nebylo tak možné přímo volat metody objektu, i když se zrovna prováděl kód v jiné z jeho metod. Toto se stávalo hlavně při obsluze asynchronního volání, kdy si metoda musela získat svého vlastníka z globálního kontejneru, kam byly všechny velké třídy (Application, Sync, Viewer a Model) ukládány.

Neduhem aplikací napsaných v JavaScriptu je také příliš mnoho funkcí, které je nutné zakládat velmi často, a celý kód se tak znepřehledňuje kvůli velkému množství závorek. Tento problém by částečně mohla vyřešit knihovna CoffeeScript\cite{coffeescript}, která používá hlavně odsazování a spousty závorek nepotřebuje, protože je schopná si je \uv{domyslet}. Bohužel je určena hlavně pro Linux a její zprovoznění na Windows se ukázalo jako velmi problematické. Také by to znamenalo nutnost učit se novou syntaxi.

\chapter{Závěr}

Výsledkem této práce je nástroj, který v sobě integruje správu tří verzovacích serverů. Protože se poskytovaná API hodně liší, byla nutná spousta kompromisů pro zachování konzistence ovládání aplikace. Tím je sice uživatel ochuzen o několik extra funkcí poskytovaných těmito servery, ale to hlavní je v aplikaci umožněno. Tato práce nekladla důraz na vizuální stránku, která dost často rozhoduje o úspěchu aplikace.

Důležitou součástí vývoje software je testování ve všech různých podobách. Při vývoji tohoto nástroje to nebylo jinak. Během vývoje byly prováděny jednotkové (unit) testy, po dokončení implementační fáze bylo provedeno zátěžové testování výpisu úkolů, což se ukázalo jako časově velmi náročná operace. Úplně nakonec byly provedeny akceptační testy na~základě sepsaných user stories. Všechny provedené testy potvrdily funkčnost aplikace a jako taková je připravená k používání reálnými uživateli.